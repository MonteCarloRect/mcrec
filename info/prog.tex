\section{Программирование}

Хранение данных:
Отдельный массив для координат молекул  Массив[номер тарелки][номер молекулы]
Массив для координат атомов Массив[номер тарелки][номер молекулы][номер атома] --- координаты атомов хранятся относительно центров молекулы. 

массив номеров молекул в жидкости Массив[номер тарелки][максимум молекул]
массив номеров молекул в газе Массив[номер тарелки][максимум молекул]

\subsection{Перемещение из жидкости в газ или наоборот}
\begin{itemize}
	\item Выбираем случайную молекулу
	\item Рассчитываем энергию в текущем объеме
	\item Выбираем случайные координаты в другом объеме и рассчитываем энергию
	\item При успешном переносе молекулы за место текущего индекса подставляем последний и уменьшаем количество молекул на одну
	\item новые координаты вставляем в конец массива и добавляем количество молекул
\end{itemize}

\subsection{Перемещение молекулы в текущем объеме}
Изменяем только координаты в массиве молекулы

\subsection{Вращение молекулы}
При вращении изменяется только координаты в массиве для атомов