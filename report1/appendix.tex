\appendix
\section{Приложение}
% the \\ insures the section title is centered below the phrase: AppendixA

Файлы проекта с описанием содержания (представлены только основные файлы проекта):
\begin{itemize}
	\item build --- тестовая сборка для отладки
	\begin{itemize}
		\item A.gro --- файл координат атомов в молекуле
		\item data.mcr --- файл с параметрами моделирования
		\item B.gro  --- файл координат атомов в молекуле 
		\item data.top --- файл топологии молекул
		
	\end{itemize}
	\item doc --- документация по проекту
	\item initial --- моделирование входящих потоков
	\begin{itemize}
		\item data\_from\_device.cu --- перенос данных из видеокарты в хост
		\item device\_prop.cu --- определение свойств видеокарты
		\item initial\_flows.cu --- задание начальной конфигурации молекул во входящем потоке
		\item read\_gro.cu  --- чтение gro файлов
		\item read\_top.cu --- чтение top файлов
		\item data\_to\_device.cu 	--- перенос данных из хоста в видеокарту
		\item free\_arrays.cu --- освобождение памяти на видеокарте
		\item rcut.cu --- вычисление добавки, связанной с обрезанием потенциала взаимодействия
		\item read\_options.cu --- чтение параметров моделирования из файла 
		\item single\_box.cu --- основной алгоритм метода Монте-Карло
	\end{itemize}
	\begin{itemize}
		\item dbox\_init.cu --- вычисление начального состояния системы
		\item double\_devtohost.cu --- подготовка данных и перенос данных на видеокарту
		\item double\_step.cu --- ядро вычислений на видеокарте
	\end{itemize}
	\item write --- содержит файлы с выводом логов и выходных данных
	\item CMakeLists.txt --- конфигурация для сборки Cmake
	\item global.h --- глобальные константы
	\item initial.h --- инициализация глобальных переменных программы
	\item mcrec.cu --- основной файл программы (начало программы)
	\item mcrec.h --- объявление новых типов структур
		
\end{itemize}
