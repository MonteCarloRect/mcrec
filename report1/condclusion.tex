\addcontentsline{toc}{section}{Заключение}
\section*{Заключение}
В работе предложен метод расчета процесса ректификации с использованием молекулярно-статистических методов. Метод базируется на следующих допущениях:
\begin{enumerate}
	\item количество молекул на тарелке рассчитывается исходя их входящих на тарелку потоков жидкой фазы с верхней тарелки, и газовой фазы нижней тарелки;
	\item на тарелку питания и с верхней и нижней тарелки добавляется и удаляются заданные количество молекул;
	\item суммарная энергия молекул на тарелке высчитывается исходя из материальных потоков пунктов 1 и 2;
	\item можно задавать дополнительный подвод (например в кубе колонны) или отвод энергии;
	\item на тарелках устанавливаются условия фазового равновесия, рассчитываемые методом Монте-Карло с использованием ансамбля Гиббса, температура и соответствующие условия равновесия определяются рассчитанной полной энергией молекул на тарелке.
\end{enumerate}

Таким образом соблюдаются законы сохранения массы и энергии. Достоинством описанного метода является то, что для расчета процесса ректификации нет необходимости в экспериментальных или полуэмпирических (выражений коэффициентов активностей и давления насыщенных паров чистых компонентов) данных по условиям фазового равновесия. Метод позволяет рассчитывать периодические и непрерывные процессы с различным количеством входящих и выходящих потоков как массы так и энергии.

Разработанный алгоритм расчета частично реализован в виде программы. Для увеличения производительности программы использована программно-аппаратная технология CUDA, позволяющая проводить вычисления на графических процессорах. Исходный код программы выложен в репозитории github (ссылка https://github.com/MonteCarloRect/mcrec) под открытой лицензией. Сравнение энергии и давления для входящих на тарелку питания потоков леннадж-джонсовского флюида с литературными данными показало хорошее согласование данных. Это свидетельствует о верной реализации алгоритмов молекулярного моделирования.

В связи с тем, что для увеличения производительности программы программа разрабатывалась с возможностью использования всех установленных на компьютере видеокарт, написание и отладка программы существенно усложнилась. Поэтому не хватило времени для реализации блока переноса молекул между тарелками. На начало работы над проектом литературный обзор показал, что нет программ позволяющих вычислять фазовое равновесие с использованием метода ансамбля Гиббса на видеокартах. Это послужило одной из причин написания программы заново, а не использование готового кода. В настоящее время уже существуют программы с открытым исходным кодом, позволяющие рассчитывать фазовое равновесие молекулярными методами на видеокартах \cite{Nejahi2019}.Использование кусков кода данной программы может ускорить разработку. 


Для визуализации расчетов создана программа (ссылка на репозиторий \\ https://github.com/kzncvs/mc\_visualisation)позволяющая на web-станице в интерактивном режиме просматривать результаты расчетов в виде графиков зависимости свойств по высоте колонны и в зависимости от цикла работы программы. 