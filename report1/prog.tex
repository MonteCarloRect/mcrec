\section{Программная реализация}
Исходный код программы доступен в репозитории открытых проектов github  по ссылке \cite{github_mcr}. Текущая ветка разработки initialsim. В качестве системы сборки приложения используется Cmake. Для увеличения производительности программы разработка велась на языке CUDA. Библиотеки для осуществления многопоточных рачетов на центральном процессоре не осуществлялись в связи с тем, что на центральном процессоре осуществляется лишь небольшая часть вычислений, связанная с подготовкой исходных данных.

\subsection{Ввод исходных данных}
Для проведения моделирования необходимо задать исходные данные, что является первым этапом на схеме \ref{fig:alg_scheme}. Исходные данные были разделены на три типа: 
\begin{itemize}
	\item исходные параметры моделирования (количество материальных потоков, составы и т.д.) хранятся в файле <<data.mcr>>;
	\item положение атомов в каждом из типов молекул хранятся в фалах <<name.gro>> (название файла должно соответствовать данным введенным в файл <<data.mcr>>);
	\item  топология молекул (параметры внутри и межмолекулярного взаимодействия) хранятся в файле <<data.top>>.
\end{itemize}


Файл <<data.mcr>> содержит последовательно представленные данные:
\begin{itemize}
	\item количество веществ \item названия файлов с координатами атомов \item количество входящих в колонну материальных потоков \item ансамбль для расчета входящих потоков \item температура входящих потоков [K] \item плотность входящих потоков [моль/л] \item количество молекул добавляемых на тарелку в каждый цикл \item мольные доли каждого из компонентов
\end{itemize}
В дальнейшем будет реализован форматированный ввод данных после того как будут известны все параметры моделирования.

В разрабатываемой программе реализована совместимость с форматом файлов пакета gromacs (файлы разрешения <<gro>> и  <<top>>). Файлы с координатами атомов имеют вид:
\begin{lstlisting}
MD of 2 waters, t= 0.0
3
1WATER  OW1    1   0.126   1.624   1.679  0.1227 -0.0580  0.0434
1WATER  HW2    2   0.190   1.661   1.747  0.8085  0.3191 -0.7791
1WATER  HW3    3   0.177   1.568   1.613 -0.9045 -2.6469  1.3180
1.82060   1.82060   1.82060
\end{lstlisting}

Первая строка является комментарием, во второй указано количество атомов в молекуле (или во всей системе). Далее идет информация по каждому атому: номер молекулы, название молекулы, номер атома, $x$, $y$, $z$ координаты, $\upsilon_x$,  $\upsilon_y$, $\upsilon_z$ компоненты скорости. В последней строчке указаны размеры ячейки (эта строчка в данной программе не используется).

Файл топологии имеет вид:
\begin{lstlisting}
[ defaults ]
1    2 no    1.00000   1.00000
[ atomtypes ]
Si    28.0000    0.9000 A    0.38264   1.25700
C2   14.0000   -0.2250 A    0.39500   0.38244
O    16.0000   -0.4500 A    0.31181   0.62850
CE1    15.0110    0.0000 A    0.36072   0.99898
CE2    14.0110    0.2526 A    0.34612   0.71746
OE1    15.9994   -0.6971 A    0.31496   0.70717
HE1     1.0080    0.4415 A    0.01000   0.00000
[ bondtypes ]
CE1    CE2      1        0.19842   500000.00000
CE2    OE1      1        0.17158   500000.00000
OE1    HE1      1        0.09505   500000.00000
OW     HW      1        0.09572   500000.00000
OW     MW      1        0.01546   500000.00000
[ angletypes ]
CE1    CE2    OE1      1  90.95000 500.00000
CE2    OE1    HE1      1 106.36800 500.00000
HW     OW     HW       1 104.52000 500.00000
MW    OW    HW       1  52.26000 500.00000
[ dihedraltypes ]
CE1    CE2    OE1    HE1      5   5.00000   0.00000   0.00000   0.00000

[ moleculetype ]
ETH  3
[ atoms ]
1  CE1      1 ETH CE1    1   0.00000  15.01100
2  CE2      1 ETH CE2    1   0.25560  14.01100
3  OE1      1 ETH OE1    1  -0.69711  15.99940
4  HE1      1 ETH HE1    1   0.44151   1.00800
[ bonds ]
1    2    1
2    3    1
3    4    1
[ pairs ]
[ angles ]
1    2    3    1
2    3    4    1
[ dihedrals ]
1    2    3    4    5
[ exclusions ]
[ position_restraints ]
[ constraints ]

[ system ]
generateg
[ molecules ]
ETH    1000
\end{lstlisting}

В квадратных скобках указаны ключевые слова, после которых идет определенный блок данных. На текущий момент реализовано фиксированное расположение атомов в молекуле, и внутримолекулярная конфигурация молекул не изменяется. Поэтому с файла считываются только данные блока  [atomtypes]. В данном блоке в каждой строчке содержится: название атома, молекулярная масса, заряд молекулы, тип атома (в gromacs есть виртуальные атомы не обладающие массой), параметр $\sigma$, параметр $\varepsilon$. 

\subsection{Расчет свойств входящих потоков}
Для добавления на тарелки питания новых молекул необходимо знать энергию этих молекул. В случае однофазного входящего потока учитывая число степеней свободы необходимо задать покомпонентный состав, температуру и одну из двух величин: давление или плотность смеси. На текущий момент реализован $NVT$ ансамбль для расчета свойств входящего потока.



\subsubsection{Генерирование стартовой ячейки}
Для разработки использовалась видеокарта GTX 1060, поэтому максимальное количество потоков на блок составляло 1024. Количество выделяемых блоков соответствовало количеству входящих потоков. Количество выделяемых потоков на блок выделялось так, чтобы не превысить максимальное количество потоков на видеокарту. Поэтому удобнее использовать количество молекул, кратное 512, это позволяет избежать лишних проверок в каждом цикле на соответствие максимальному количеству. В данной работе использовалось 4098 молекул при расчете. 

Молекулы веществ изначально расставлялись  в случайном порядке в узлах кубической кристаллической решетки. 

Далее считанные из исходных файлов данные переносились в одномерные массивы для возможности обработки полученных данных на видеокарте. В связи с переносом данных в одномерные массивы были созданы дополнительные массивы, использующиеся для получения индекса определенной молекулы или атома.  

На текущий момент разработки в программе реализован потенциал взаимодействия Леннард~--Джонса:
\begin{equation}
	\phi(r) = 4 \varepsilon \left( \left( \dfrac{\sigma}{r} \right) ^{12} - \left( \dfrac{\sigma}{r} \right) ^6 \right)
\end{equation}
где $\varepsilon$ --- параметр, характеризующий глубину потенциальной ямы, $\sigma$ --- расстояние на котором потенциал равен нулю.  Увеличение производительности при расчете проводится за счет параллельного вычисления энергии взаимодействия. Так, при выполнении алгоритма в один поток при используемых параметрах моделирования, необходимо на каждый шаг Монте~-- Карло вычислить 4097 энергий взаимодействия. В случае выполнения этого алгоритма на видеокарте в 512 потоков, каждый поток вычисляет лишь взаимодействия с 8 молекулами. Однако, каждый поток записывает данные в заданный элемент массива энергий взаимодействия двух молекул. Поэтому далее проводится суммирование элементов массива с использованием специальных алгоритмов разработанных для много поточных систем \cite{cuda_book}.

Термодинамические свойства вычисляются каждый 20 шаг Монте~-- Карло. Измеренные вычисления усредняется по определенным блокам, в каждом блоке проводится 20~000 вычислений микросвойств системы. Далее в каждом блоке определяются средние значения термодинамических свойств. Вычисления производились с использованием 5 блоков. Каждый из блоков принимается как отдельный численный эксперимент, поэтому проводится статистическая обработка с определением среднего и стандартного отклонения.

Усреднение по блокам проводится только после установления термодинамического равновесия. Достижение термодинамического равновесия определяется  из условия, что максимальное отклонение давления и энергии в блоке не превышает 7\%. 

\subsection{Сравнение результатов расчетов с литературными данными}
Для отладки приложения и исключения ошибок в результатах, было проведено сравнение результатов работы программы с результатами других авторов. Верификация работы программы проведена на примере сферически симметричного леннард-джонсовкого флюида. 
В связи с этим в работе термодинамические величины приведены в безразмерном виде: числовая плотность $n^* = n \sigma ^3 = \frac{N}{V}$, давление $p^* = \frac{p \sigma^3}{\varepsilon}$, температура $T^* = \frac{k_B T}{\varepsilon}$, энергия $U^* = \frac{U}{\varepsilon}$, где $N$ --- количество молекулы в ячейке, $V$ --- объем ячейки.

В таблице \ref{tab.res} представлено сравнение результатов работы разработанной программы с литературными данными \cite{Johnson1993}. Среднее отклонение по энергии составляет 0.39\% (максимальное 1.09 \%) по давлению 1.15\% (максимальное 4.77 \%). Однако кроме одного значения под давлению результаты входят в доверительный интервал численного эксперимента. Хорошее согласование результатов между собой свидетельствует о том, что алгоритм реализован верно. 

\begin{table}[]
	\caption{Сравнение результатов расчетов с экспериментальными данными} \label{tab.res}
	\begin{tabular}{|l|l||l|l||l|l||l|l|}
		
		\hline
		\multirow{2}{*}{$n^*$} & \multirow{2}{*}{$n^*$} & \multicolumn{2}{l||}{Литературные данные} & \multicolumn{4}{l|}{Данная работа} \\ \cline{3-8}
		 & & $p^*$ & $U^*$ & $p^*$ & $U^*$ & $\Delta p^*$ &  $\Delta U^*$ \\ \hline
		0.5 & 5 & -2.365 & 4.654 & -2.373 & 4.587 & 0.32 & -1.44 \\
		0.1 & 2 & -0.667 & 0.1777 & -0.668 & 0.178 & 0.14 & 0.04 \\
		0.2 & 2 & -1.306 & 0.329 & -1.307 & 0.330 & 0.05 & 0.40 \\
		0.4 & 2 & -2.538 & 0.705 & -2.536 & 0.701 & -0.07 & -0.50 \\
		0.5 & 2 & -3.149 & 1.069 & -3.144 & 1.075 & -0.17 & 0.54 \\
		0.6 & 2 & -3.746 & 1.756 & -3.744 & 1.743 & -0.05 & -0.73 \\
		0.7 & 2 & -4.3 & 3.024 & -4.288 & 3.011 & -0.27 & -0.44 \\
		0.8 & 2 & -4.753 & 5.28 & -4.701 & 5.427 & -1.09 & 2.79 \\
		0.9 & 2 & -5.03 & 9.09 & -4.994 & 9.249 & -0.71 & 1.75 \\
		0.1 & 1.2 & -0.84 & 0.0776 & -0.833 & 0.078 & -0.82 & -0.06 \\
		0.1 & 1.15 & -0.869 & 0.0707 & -0.866 & 0.071 & -0.32 & 0.57 \\
		0.05 & 1 & -0.478 & 0.0369 & -0.481 & 0.037 & 0.60 & 0.19\\ 
		0.6 & 1 & -4.228 & -0.269 & -4.208 & -0.264 & -0.48 & -1.72\\
		0.8 & 1 & -5.533 & 1.03 & -5.524 & 1.043 & -0.16 & 1.28 \\
		0.9 & 1 & -6.062 & 3.24 & -6.027 & 3.395 & -0.57 & 4.77 \\ \hline
		
	\end{tabular}
\end{table}

\subsection{Сравнение производительности}

В таблице \ref{tab.time} представлено время вычисления 400~000 шагов Монте-Карло для системы содержащей 4096 молекул. В программе не реализована возможность проведения параллельных расчетов на центральном процессоре, таким образом, влияние процессора несущественно сказывается на производительности программы. Также сложно сделать однозначный вывод из-за того, что системы имеют различную операционную систему и компилятор. Сравнивая различные видеокарты можно сделать вывод, что видеокарта GTX 1080 немного опережает по производительности GTX 1060. Однако рассматривая отношение цены устройства к производительности видеокарта GTX 1060 лучше. Также стоит отметить, что при запуске программы не северных операционных системах, на видеокартах GTX 1060 наблюдались задержки работы графического окружения.

\begin{table}[]
	\caption{Время вычисления на компьютерах с различной конфигурацией} \label{tab.time}
	\begin{tabular}{|c|c|}
		\hline
		Конфигурация компьютера & Время вычисления, с \\ \hline
		OS: ubuntu server 18.04 / gcc 6 &\\		
		Процессор: AMD FX(tm)-6200& 43\\
		Видеокарта: GTX 1060 & \\ \hline
		
		OS: ubuntu desktop 16.04 / gcc 5 &\\
		Процессор: AMD Phenom(tm) II X6& 41\\
		Видеокарта: GTX 1060 & \\ \hline
		
		OS: ubuntu server 16.04 / gcc 5 &\\
		Процессор: AMD Phenom(tm) II X4&		40\\
		Видеокарта: GTX 1080 & \\ \hline
		
		OS: ubuntu desktop 18/04 / gcc 6 &\\
		Процессор: AMD Ryzen 7 1700 & 	43\\
		Видеокарта: GTX 1060 & \\ \hline
		
	\end{tabular}
\end{table}





