\begin{center}
	\textsc{\large{Министерство науки и высшего образования Российской Федерации}\\
	\footnotesize{Федеральное государственное бюджетное образовательное учреждение высшего образования}\\ 
	\small{\textbf{«Казанский национальный исследовательский технологический университет»}}\\}
\end{center}

	\hfill \break
	\normalsize{УДК: 544.272:66.011}\\
	\normalsize{№ Госрегистрации \textbf{АААА-А18-118032690262-8}}\\
	\normalsize{Инв. №}\\
	
	\large
	~\hspace{9cm}\textsc{Утверждаю}
	
	~\hspace{7cm} Проректор по научной работе
	
	~\hspace{7cm}\underline{\hspace{3cm}} Сабирзянов А.Н.
	
	~\hspace{7cm}<<\underline{\hspace{1cm}}>> \underline{\hspace{4cm}} г.
	
	\hfill \break
\begin{center}
	\Large{Отчет о научно~-- исследовательской работе}\\
	\hfill \break
	\Large{по теме: Молекулярно~-- статистическое моделирование процесса ректификации\\(промежуточный)}\\
	\hfill \break
	\hfill \break
	\normalsize
	\large
Руководитель темы:\hspace{5cm}   \underline{\hspace{2.7cm}} Анашкин И.П.
\hfill \break
\vspace{8cm}
\hfill \break
 Казань 2019
\end{center}
\thispagestyle{empty} % выключаем отображение номера для этой страницы


\newpage

\section*{Список исполнителей:}
\normalsize{ 
	\begin{tabular}{cccc}
		Руководитель проекта & \hspace{3cm} &  \underline{\hspace{3cm}} & Анашкин И.П. \\\\
		\hfill \break
		Исполнители: &  & \underline{\hspace{3cm}} & Казанцев С.В. \\\\
	\end{tabular}
\thispagestyle{empty} % выключаем отображение номера для этой страницы

\newpage
\section*{Реферат}
страниц -- \TotalValue{totalpages}

рисунков -- \TotalValue{totalfigures}

таблиц -- \TotalValue{totaltables}

приложений -- 1 %\TotalValue{appendixchapters}

Ключевые слова: ректификация, метод Монте-Карло, межмолекулярное взаимодействие
\\

В проекте разработан молекулярно-статистический метод моделирования процесса ректификации. В основе метода лежит комбинация трех подходов: теоретических тарелок, метода Монте~-- Карло и законов сохранения. В основе алгоритма лежит циклическое выполнение следующих этапов: добавление молекул на тарелку питания и удаление молекул из куба колонны и верхней тарелки; расчет фазового равновесия на каждой тарелке; перемещение газовой фазы на тарелку выше, жидкой фазы на тарелку ниже; пересчет энергий и температур на тарелке.

Разработана программа, реализующая некоторые блоки предложенного алгоритма: ввод исходных данных, расчет свойств входящих потоков, входящих на тарелки питания. Программа написана с использованием программно-аппаратной технологии CUDA, исходный текст программы открыт и размещен в репозитории открытых проектов. Файлы с исходными данными программы совместимы с программным пакетом gromacs. Совместимость форматов позволяет использовать множество созданных ранее файлов с различными наборами параметров межмолекулярного взаимодействия.

Были проведены тестовые расчеты свойств входящих на тарелки потоков на примере модельного флюида, межмолекулярное взаимодействие которого описывается потенциалом Леннард-Джонса. Результаты расчетов совпадают с литературными данными, что свидетельствует о верной работе алгоритма.

Проведены расчеты типовой задачи на компьютерах с различной конфигурацией с целью определения времени расчета. 
\thispagestyle{empty}
\newpage

\tableofcontents

\thispagestyle{empty}

\newpage
