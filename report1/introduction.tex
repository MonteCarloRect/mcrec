\addcontentsline{toc}{section}{Введение}
\subsection*{Введение}

Ректификация --- один из наиболее распространенных процессов разделения смесей, применяемых в химической промышленности. Для расчета процесса ректификации разработано множество подходов, в основе которых лежит использование данных о фазовом равновесии компонентов разделяемой смеси. Поэтому, расчетную схему ректификационной колонны необходимо дополнить моделями для описания давления насыщенных паров чистых компонентов и коэффициентов активности в смеси. Однако экспериментальные данные доступны не для всех комбинаций веществ, или доступны в ограниченной области состояний. Решением может выступать использование методов групповых составляющих \cite{Skjold-Jorgensen1979,Tiegs1987,Wittig2003}, однако данные методы не гарантируют хорошего согласования с экспериментальными данными.

В данной работе предлагается использование молекулярно-статистических методов для моделирования процесса ректификации. В основе предлагаемого подхода лежит использование законов сохранения, метода теоретических тарелок и метода ансамбля Гиббса для вычисления фазового равновесия.
